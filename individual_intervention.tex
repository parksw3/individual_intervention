\documentclass[12pt]{article}
\usepackage[top=1in,left=1in, right = 1in, footskip=1in]{geometry}

\usepackage{graphicx}
\usepackage{xspace}
%\usepackage{adjustbox}

\newcommand{\comment}{\showcomment}
%% \newcommand{\comment}{\nocomment}

\newcommand{\showcomment}[3]{\textcolor{#1}{\textbf{[#2: }\textsl{#3}\textbf{]}}}
\newcommand{\nocomment}[3]{}

\newcommand{\jd}[1]{\comment{cyan}{JD}{#1}}
\newcommand{\swp}[1]{\comment{magenta}{SWP}{#1}}
\newcommand{\bmb}[1]{\comment{blue}{BMB}{#1}}
\newcommand{\djde}[1]{\comment{red}{DJDE}{#1}}

\newcommand{\eref}[1]{Eq.~\ref{eq:#1}}
\newcommand{\fref}[1]{Fig.~\ref{fig:#1}}
\newcommand{\Fref}[1]{Fig.~\ref{fig:#1}}
\newcommand{\sref}[1]{Sec.~\ref{#1}}
\newcommand{\frange}[2]{Fig.~\ref{fig:#1}--\ref{fig:#2}}
\newcommand{\tref}[1]{Table~\ref{tab:#1}}
\newcommand{\tlab}[1]{\label{tab:#1}}
\newcommand{\seminar}{SE\mbox{$^m$}I\mbox{$^n$}R}

\usepackage{amsthm}
\usepackage{amsmath}
\usepackage{amssymb}
\usepackage{amsfonts}

\usepackage{lineno}
\linenumbers

\usepackage[pdfencoding=auto, psdextra]{hyperref}

\usepackage{natbib}
\bibliographystyle{chicago}
\date{\today}

\usepackage{xspace}
\newcommand*{\ie}{i.e.\@\xspace}

\usepackage{color}

\newcommand{\Rx}[1]{\ensuremath{{\mathcal R}_{#1}}\xspace} 
\newcommand{\Ro}{\Rx{0}}
\newcommand{\Rc}{\Rx{\mathrm{c}}}
\newcommand{\RR}{\ensuremath{{\mathcal R}}\xspace}
\newcommand{\Rhat}{\ensuremath{{\hat\RR}}}
\newcommand{\Rnaive}{\ensuremath{{\mathcal R}_{\textrm{\tiny naive}}}\xspace}
\newcommand{\tsub}[2]{#1_{{\textrm{\tiny #2}}}}
\newcommand{\dd}[1]{\ensuremath{\, \mathrm{d}#1}}
\newcommand{\dtau}{\dd{\tau}}
\newcommand{\dx}{\dd{x}}
\newcommand{\dsigma}{\dd{\sigma}}

\newcommand{\tstart}{\ensuremath{\tsub{t}{start}}\xspace}
\newcommand{\tend}{\ensuremath{\tsub{t}{end}}\xspace}

\newcommand{\betaeff}{\ensuremath{\tsub{\beta}{eff}}\xspace}
\newcommand{\Keff}{\ensuremath{\tsub{K}{eff}}\xspace}

\newcommand{\pt}{p} %% primary time
\newcommand{\st}{s} %% secondary time

\newcommand{\psize}{{\mathcal P}} %% primary cohort size
\newcommand{\ssize}{{\mathcal S}} %% secondary cohort size

\newcommand{\gtime}{\sigma} %% generation interval
\newcommand{\gdist}{g} %% generation-interval distribution

\newcommand{\geff}{g_{\textrm{eff}}} %% generation-interval distribution

\newcommand{\total}{{\mathcal T}} %% total number of serial intervals

\begin{document}

\begin{flushleft}{
	\Large
	\textbf\newline{
		Quantifying the effects of population- and individual-based intervention strategies
	}
}
\newline
\\
Sang Woo Park\textsuperscript{1,*}
\\
\bigskip
\textbf{1} Department of Ecology and Evolutionary Biology, Princeton University, Princeton, NJ, USA
\\
\bigskip

*Corresponding author: swp2@princeton.edu
\end{flushleft}


\section{Introduction}

\section{Methods}

\subsection{Renewal equation framework}

We begin by describing the uncontrolled spread of infection using the renewal equation framework.
Let $K(\tau)$ represent the intrinsic infection kernel, defined as the rate at which infectious contacts are made by an average infected individual infected $\tau$ time units ago in the absence of any internvetion.
The integral of $K(\tau)$, representing the total infectiousness of an average infected individual, corresponds to the basic reproduction number: $\Ro = \int K(\tau) \dtau$.
The kernel, normalized by the total infectiousness, corresponds to the intrinsic generation-interval distribution: $g(\tau) = K(\tau)/\Ro$.
The generation interval, defined as the time between when an individual is infected and when that individual infects another person, characterizes the time scale of disease transmission.
Then, the rate at which an individaul infected at time $t$ will generate a secondary case $\tau$ time units after their infection (i.e., the effective infection kernel, $\Keff(t,\tau)$) can be written as a product between the proportion susceptible in $S$ and infection kernel $K$:
\begin{equation}
\Keff(t,\tau) = S(t+\tau) K(\tau).
\end{equation}
Ignoring births and deaths, we can further express the dynamics of the proportion susceptible $S(t)$ and incidence $i(t)$ in the absence of intervention as follows:
\begin{align}
\frac{\mathrm{d}S}{\mathrm{d}t} &= - i(t),\\
i(t) &= \int_0^\infty i(t-\tau) \Keff(t-\tau, \tau) \dtau = S(t) \int_0^\infty i(t-\tau) K(\tau) \dtau
\end{align}
This model, also known as renewal equations, generalizes the dynamics of many compartmental models.

\subsection{Population- and individual-based intervention}

\begin{figure}[!th]
\includegraphics[width=0.45\textwidth]{pop_ind_compare1.pdf}
\mbox{\hspace{0.01\textwidth}}
\includegraphics[width=0.45\textwidth]{pop_ind_compare2.pdf}
\\
\includegraphics[width=0.45\textwidth]{pop_ind_compare3.pdf}
\mbox{\hspace{0.01\textwidth}}
\includegraphics[width=0.45\textwidth]{pop_ind_compare4.pdf}
\caption{
\textbf{I'm a caption}
}
\label{fig:indpop}
\end{figure}

To model the impact of intervention during an ongoing epidemic, we first distinguish population-based interventions, which reduce the transmission potential of all infected individuals by a constant factor regardless of their time of infection, from individual-based interventions, which depend on the time of infection.
Population-based interventions can be thought of as interventions that reduce the transmission rate and include social distancing and school closures.
Individual-based interventions can be thought of as interventions that reduce the duration of infectious periods and include case identification and isolation.
Some interventions, like contact tracing, depend not only on the infection time of an infected individual but also on their infector's infection time, and therefore do not belong to either population- or individual-based interventions;
for simplicity, we do not consider these at this stage.

Let $P(t)$ represent a population-level intervention that modulates the infection kernel multicatively such that $P(t)=1$ corresponds to intervention that has no effect and $P(t) < 1$ corresponds to intervention that reduces transmission.
Then, the effective infection kernel of an individual infected at time $t$ under a population-level intervention can be written as:
\begin{equation}
\Keff(t, \tau) = S(t+\tau) P(t + \tau) K(\tau).
\end{equation}
For example, a social distancing measure that reduces transmission by a factor of $1/\phi$ between time \tstart and \tend can be modeled as:
\begin{equation}
P(t) = \begin{cases}
1 & t < \tstart\\
\phi & \tstart \leq t < \tend\\
1 & \tend \leq t
\end{cases}.
\end{equation}
Such intervention reduces the infectiousness of individuals who are infected after \tstart by a constant amount throughout the course of their infection (\fref{indpop}A).


Population-level internvention can take effect immediately:
when lockdown is imposed at time \tstart, overall transmission rate decreases immdiately (\fref{indpop}C).
Population-level internvention is also strength-like \swp{because ...}.

Then, given population- and individual-based interventions, $P(t)$ and $I(t, t')$, that modulate the infection kernel multicatively, the dynamics of $S(t)$ and $i(t)$ can be now written as:
\begin{equation}
\begin{aligned}
\frac{\mathrm{d}S}{\mathrm{d}t} &= - i(t),\\
i(t) &= S(t) P(t) \int_0^\infty i(t-\tau) I(t,t-\tau) \beta(\tau) \dtau.
\end{aligned}
\end{equation}

Individual-based intervention $I(t, t')$, such as case isolation, targets each infected individual and therefore depends on calendar time $t$ as well as time of infection $t'$.
In particular, given hazard $h(\tau)$ of being isolated that depends on time since infection $\tau$, the probability that an individual infected at time $t'$ has not been isolated by time $t$ given that the individual-based intervention takes place between time \tstart and \tend can be modeled as:
\begin{equation}
I(t, t') = \begin{cases}
1 & t < \tstart\\
\exp\left(-\int_{\tstart-t'}^{t-t'} h(s) \dd{s}\right) & \tstart \leq t < \tend, t' < \tstart \\
\exp\left(-\int_{0}^{t-t'} h(s) \dd{s}\right) & \tstart \leq t < \tend, \tstart \leq t' < \tend\\
\exp\left(-\int_{\tstart-t'}^{\tend-t'} h(s) \dd{s}\right) & \tend < t, t' < \tstart \\
\exp\left(-\int_{0}^{\tend-t'} h(s) \dd{s}\right) & \tend < t, \tstart \leq t' < \tend\\
1 & \tend < t, \tend < t'
\end{cases}
\end{equation}
Unlike the population-level intervention, individual-level internvetion does not take effect immediately.
Individual-level intervention is also speed-like \swp{because ...}.
For simplicity, we do not consider individual-based intervention that depends on infection of other individuals, such as contact tracing, at this stage.



\subsection{Quantifying changes in the reproduction number}

In order to quantify the effect of intervention, we want to compare how the intervention reduces the number of new cases beginning from the day of its implementation.
We argue that this has to be done by looking backward from the time of infection: by comparing the ratio of current incidence at time $t$ with previous incidence at time $t-\tau$ multiplied by their current infectiousness at time $t$.
This ratio is often referred to as the instantaneous reproduction number $\RR(t)$.

In order to define the instantaneous reproduction number $\RR(t)$, we want to account for changes in the underlying generation-interval distribution due to intervention.
Thus, we define effective generation-interval distribution $\geff(t, \tau)$ which accounts for the individual-level intervention.
Given time of infection $t'$ and time since infection $\tau$, we have:
\begin{equation}
\geff(t', \tau)= \frac{\beta(\tau)I(t'+\tau,t')}{\int_0^\infty \beta(x)I(t'+x,t') \dx}.
\end{equation}
Then, we can rewrite Eq~?? as:
\begin{equation}
\begin{aligned}
\frac{\mathrm{d}S}{\mathrm{d}t} &= - i(t),\\
i(t) &=  \mathcal R(t) \int_0^\infty i(t-\tau)  \geff(t-\tau, \tau) \dtau.
\end{aligned}
\end{equation}
Here, the instantaneous reproduction number $\RR(t)$ is defined as:
\begin{align}
\RR(t) &= S(t) P(t) \int_0^\infty I(t,t-\tau) \beta(\tau) d\tau,\\
&= \Ro S(t) P(t) \int_0^\infty I(t,t-\tau) g(\tau) d\tau,
\end{align}
which depends on the proportion of the susceptible population and proportional reduction in the intrinsic infectivity of an individual, weighted by the intrinsic generation-interval distribution.
Then, we can estimate the instantaneous reproduction number as follows:
\begin{equation}
\RR(t)= \frac{i(t)}{\int_0^\infty i(t-\tau) \geff(t-\tau, \tau) \dtau}.
\end{equation}
This definition of the instantaneous reproduction number contrasts with the widely used definition that assumes that the underlying generation-interval distribution remains constant.

Given incidence between time $0$ and $t-\epsilon$ for $\epsilon > 0$, we can also calculate the proportional reduction $p(t)$ in true incidence $i(t)$ at time $t$ compared to incidence that we would get in the absence of intervention:
\begin{equation}
p(t) = \frac{S(t) P(t) \int_0^\infty i(t-\tau) I(t,t-\tau) \beta(\tau) \dtau}{S(t) \int_0^\infty i(t-\tau) \beta(\tau) \dtau}.
\end{equation}
By substituting true incidence $i(t)$ in the numerator and rearranging, we get a familiar expression for the estimator for $\RR(t)$, which we refer to as $\tsub{\RR}{prop}(t)$ that depends on the proportion of the susceptible population and proportional reduction in incidence:
\begin{equation}
\tsub{\RR}{prop}(t) = \Ro S(t) p(t) = \frac{i(t)}{\int_0^\infty i(t-\tau) g(\tau) \dtau}.
\end{equation}

Finally, several studies have considered `forward-looking' reproduction number $\Rc(t)$, defined as the average number of secondary cases caused by an individual infected at time $t$.
This value has been referred to as the case reproduction number.
In our case, the case reproduction number can be estimated as:
\begin{equation}
\begin{aligned}
\Rc(t) &= \int_0^\infty \beta(\tau) S(t+\tau) P(t+\tau) I(t+\tau, t) \dtau,\\
&= \int_0^\infty \RR(t+\tau) \geff(t+\tau, \tau) \dtau.
\end{aligned}
\end{equation}
Therefore, the case reproduction number can be only calculated retrospectively, even though it is `forward-looking' by definition.

\subsection{Using serial-interval distribution instead of generation-interval distribution}






\end{document}
